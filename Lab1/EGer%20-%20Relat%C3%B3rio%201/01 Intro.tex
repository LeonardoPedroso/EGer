\section{Introdução}

No âmbito da unidade curricular de Electrónica Geral, desenvolveu-se o presente relatório acerca do primeiro trabalho de laboratório. Esta atividade dividiu-se em duas sessões.

Na primeira, duas montagens de filtros ativos formadas por circuitos elementares são realizadas e analisadas, retirando-se conclusões acerca do caraterísticas de resposta teóricas de cada uma, das alterações que ocorrem a nível experimental e de pequenas mudanças que se podem fazer aos circuitos, utilização de um divisor de tensão e de uma resistência variável.

Na segunda sessão, tenta-se implementar parte de um detetor de proximidade: o emissor e o filtro passa-banda para seleção da frequência emitida. O emissor é constituído por um oscilador que é o principal objetivo de estudo da sessão. Por forma a simular o funcionamento do sistema, simula-se a distância a um obstáculo através de um circuito atenuador. Sugerem-se também alterações para melhorar o desempenho do sistema.

\clearpage