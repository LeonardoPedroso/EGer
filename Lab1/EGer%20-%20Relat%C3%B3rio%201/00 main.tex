\documentclass[a4paper, oneside]{article}

%% Language and font encodings
\usepackage[portuguese]{babel}
\usepackage[utf8x]{inputenc}
\usepackage[T1]{fontenc}



%% Sets page size and margins
\usepackage[a4paper,top=2.2cm,bottom=2.2cm,left=2cm,right=2cm,marginparwidth=1.75cm]{geometry}

%% Useful packages
\usepackage{amsmath}
\usepackage{graphicx}


\usepackage{fancyhdr}
\usepackage{titlepic}
\usepackage{tabto}
\usepackage{amsfonts}
\usepackage{commath}
\usepackage{float}
\usepackage{setspace}
\usepackage[labelfont=bf]{caption}
\usepackage{subcaption}
\usepackage[table,xcdraw]{xcolor}
\usepackage{subfiles}
\usepackage{multicol}
\usepackage{comment}
\usepackage{color}
\usepackage{mathtools}
%\usepackage[dvipsnames]{xcolor}
\usepackage{lastpage}
\usepackage{indentfirst}
\usepackage{titlesec}
\usepackage{siunitx}
\usepackage{multirow}
\usepackage{enumitem}
\usepackage{textcomp}
\usepackage{gensymb}
\usepackage{systeme}
\usepackage{wrapfig}
\usepackage[numbered,framed]{matlab-prettifier}
\usepackage{hyperref}

\let\ph\mlplaceholder % shorter macro
\lstMakeShortInline"

\lstset{
  style              = Matlab-editor,
  basicstyle         = \mlttfamily,
  escapechar         = ",
  mlshowsectionrules = true,
}

\titleformat{\paragraph}
{\normalfont\normalsize\bfseries}{\theparagraph}{1em}{}
\titlespacing*{\paragraph}
{0pt}{3.25ex plus 1ex minus .2ex}{1.5ex plus .2ex}

\renewcommand{\baselinestretch}{1.2}

%\hypersetup{colorlinks,citecolor=black,filecolor=black,linkcolor=black,urlcolor=black} 

\pagestyle{fancy}
\fancyhf{}
\rhead{Filtros Activos e Osciladores}
\lhead{Electrónica Geral}
\cfoot{Página  \thepage \hspace{1pt} de \pageref{LastPage}}
\pagenumbering{roman}


\begin{document}

\begin{titlepage}
	\begin{center}
		\begin{figure}[htb!]
			\begin{center}
				\includegraphics[width=10cm]{Imagens/istlogo.jpg}
			\end{center}
		\end{figure}
        
        \begin{center}
        \LARGE{\textbf{\center Instituto Superior Técnico}}\\
        \vspace{20pt}
        \Large{\center Mestrado Integrado em Engenharia Aeroespacial}\\
        \Large{\center Electrónica Geral}\\
        \Large{\center 1º Semestre 2020/2021}\\
        \end{center}
            
        \vspace{60pt}
        \noindent\rule{15cm}{1pt}\\
         \Huge{\center \textbf{1º Trabalho de Laboratório}} \par \Huge{\center \textbf{Filtros Activos e Osciladores
}}\\
         
         \noindent\rule{15cm}{1pt}\\
        
        \vspace{60pt}
        
        
        \begin{minipage}{0.5\textwidth}
		\begin{flushleft} \large
			\textbf{Grupo 7:}\\
			89652, Carolina Pinheiro\\
			89683, José Neves \\
			89691, Leonardo Pedroso\\
		\end{flushleft}
	\end{minipage}
	~
	\begin{minipage}[b]{0.4\textwidth}
		\begin{flushright} \large
        	\textbf{Docente:} \\
            Prof. José António Beltran Gerald \\
		\end{flushright}
	\end{minipage}\\[2cm]
        
         \vspace{10pt}

        \large{Outubro de 2020}\\
        
	\end{center}
\end{titlepage}

\newpage
\renewcommand{\contentsname}{Índice}
\tableofcontents
\thispagestyle{empty}

\newpage
\pagenumbering{arabic}
\setcounter{page}{1}

\subfile{01 Intro.tex}
\subfile{02 KNH.tex}
\subfile{03 TT.tex}
\subfile{04 Oscilador.tex}
\subfile{05 Filtro Rauch.tex}
\end{document}